# Placeholder for 03_main_discussion_no.md section
Innenfor NISTs rammeverk bør cybersikkerhetstiltak skreddersys til organisasjonens spesifikke risikoprofil. For eksempel gir SP 800-53 et omfattende sett med kontroller — alt fra system- og kommunikasjonsbeskyttelse til vedlikehold og hendelseshåndtering — som kan velges basert på miljøfaktorer, trusselnivå og overordnede mål. I tillegg understreker NIST SP 800-137 viktigheten av kontinuerlig vurdering og løpende overvåking for å sikre at kontrollene til enhver tid er effektive mot nye trusler.

En vellykket innføring av NISTs retningslinjer krever ofte tett samarbeid på tvers av avdelinger. Ved å involvere nøkkelpersoner fra drift, jus og toppledelse oppnår man en kultur der sikkerhet er et felles ansvar. Gjennom å anvende NISTs prinsipper i både tekniske og organisatoriske prosesser, får man et strukturert veikart for å beskytte sensitiv informasjon, opprettholde systemintegritet og styrke tilliten hos kunder og samarbeidspartnere.