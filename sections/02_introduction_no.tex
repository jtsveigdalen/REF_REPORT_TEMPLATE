# Placeholder for 02_introduction_no.md section
National Institute of Standards and Technology (NIST) gir omfattende retningslinjer for å hjelpe organisasjoner med å etablere og opprettholde robuste cybersikkerhetsprogrammer. Disse rammeverkene og beste praksisene dekker alt fra tilgangskontroll til hendelseshåndtering, og sikrer at sikkerhetstiltakene er i samsvar med virksomhetens målsettinger og juridiske krav. Ved å ta i bruk NISTs anbefalinger, kan organisasjoner på en mer effektiv måte identifisere, vurdere og styre risikoer knyttet til informasjonssystemer og driftsprosesser.

Spesielt legger NISTs Risk Management Framework (RMF) vekt på en helhetlig tilnærming, slik at sikkerhets- og personvernkrav inkluderes helt fra planleggingsfasen og fram til systemer fases ut. Denne tilnærmingen gjør det mulig for organisasjoner å håndtere dagens trusler proaktivt, tilpasse seg nye teknologier og beskytte kritiske ressurser, samtidig som de oppfyller relevante regelverkskrav.